%%%%%%%%%%%%%%
%% Run LaTeX on this file several times to get Table of Contents,
%% cross-references, and citations.

%% w-bktmpl.tex. Current Version: Feb 16, 2012
%%%%%%%%%%%%%%%%%%%%%%%%%%%%%%%%%%%%%%%%%%%%%%%%%%%%%%%%%%%%%%%%
%
%  Template file for
%  Wiley Book Style, Design No.: SD 001B, 7x10
%  Wiley Book Style, Design No.: SD 004B, 6x9
%
%  Prepared by Amy Hendrickson, TeXnology Inc.
%  http://www.texnology.com
%%%%%%%%%%%%%%%%%%%%%%%%%%%%%%%%%%%%%%%%%%%%%%%%%%%%%%%%%%%%%%%%

%%%%%%%%%%%%%%%%%%%%%%%%%%%%%%%%%%%%%%%%%%%%%%%%%%%%%%%%%%%%%%%%
%% Class File

%% For default 7 x 10 trim size:
%\documentclass{WileySev}

%% Or, for 6 x 9 trim size
\documentclass{WileySix}

%%%%%%%%%%%%%%%%%%%%%%%%%%%%%%%%%%%%%%%%%%%%%%%%%%%%%%%%%%%%%%%%
%% Post Script Font File

% For PostScript text
% If you have font problems, you may edit the w-bookps.sty file
% to customize the font names to match those on your system.

\usepackage{w-bookps}
\usepackage{hyperref}
\usepackage{biblatex}
\usepackage{verbatim}
%\bibliographystyle{ieeetr}
\bibliography{jon.bib}
%\addbibresource{jon.bib}

%%%%%%%
%% For times math: However, this package disables bold math (!)
%% \mathbf{x} will still work, but you will not have bold math
%% in section heads or chapter titles. If you don't use math
%% in those environments, mathptmx might be a good choice.

% \usepackage{mathptmx}


%%%%%%%%%%%%%%%%%%%%%%%%%%%%%%%%%%%%%%%%%%%%%%%%%%%%%%%%%%%%%%%%
%% Graphicx.sty for Including PostScript .eps files
\usepackage{amsmath}
\usepackage{forest}
\usepackage{multirow}
\usepackage{graphicx}
\usepackage{gensymb}
\usepackage{courier}
\usepackage{hyperref}
\hypersetup{
    colorlinks=true,
    linkcolor=blue,
    filecolor=magenta,      
    urlcolor=cyan,
}
 
\urlstyle{same}

\usepackage[]{appendix}
\usepackage{listings}
\usepackage{color}
\definecolor{dkgreen}{rgb}{0,0.6,0}
\definecolor{gray}{rgb}{0.5,0.5,0.5}
\definecolor{mauve}{rgb}{0.58,0,0.82}

\lstset{frame=tb,
  language=Python,
  aboveskip=3mm,
  belowskip=3mm,
  showstringspaces=false,
  columns=flexible,
  basicstyle={\tiny\ttfamily},
  numbers=none,
  numberstyle=\tiny\color{gray},
  keywordstyle=\color{blue},
  commentstyle=\color{dkgreen},
  stringstyle=\color{mauve},
  breaklines=true,
  breakatwhitespace=true,
  tabsize=3
}

%%%%%%%%%%%%%%%%%%%%%%%%%%%%%%%%%%%%%%%%%%%%%%%%%%%%%%%%%%%%%%%%
%% Other packages you might want to use:

% for chapter bibliography made with BibTeX
% \usepackage{chapterbib}

% for multiple indices
% \usepackage{multind}

% for answers to problems
% \usepackage{answers}

%%%%%%%%%%%%%%%%%%%%%%%%%%%%%%%%%%%%%%%%%%%%%%%%%%%%%%%%%%%%%%%%
%% Change options here if you want:
%%
%% How many levels of section head would you like numbered?
%% 0= no section numbers, 1= section, 2= subsection, 3= subsubsection
%%==>>
\setcounter{secnumdepth}{3}

%% How many levels of section head would you like to appear in the
%% Table of Contents?
%% 0= chapter titles, 1= section titles, 2= subsection titles, 
%% 3= subsubsection titles.
%%==>>
\setcounter{tocdepth}{2}

%% Cropmarks? good for final page makeup
%% \docropmarks %% turn cropmarks on

%%%%%%%%%%%%%%%%%%%%%%%%%%%%%%%%%%%%%%%%%%%%%%%%%%%%%%%%%%%%%%%%
%% DRAFT
%
% Uncomment to get double spacing between lines, current date and time
% printed at bottom of page.
% \draft
% (If you want to keep tables from becoming double spaced also uncomment
% this):
% \renewcommand{\arraystretch}{0.6}
%%%%%%%%%%%%%%%%%%%%%%%%%%%%%%

\begin{document}

%%%%%%%%%%%%%%%%%%%%%%%%%%%%%%%%%%%%%%%%%%%%%%%%%%%%%%%%%%%%%%%%
%% Title Pages
%%
%% Wiley will provide title and copyright page, but you can make
%% your own titlepages if you'd like anyway

%% Setting up title pages, type in the appropriate names here:
\booktitle{PyFinances}
\subtitle{PyFinances Requirements Document}

\author{Jonathan A. Webb}
%or
%\authors{}

%% \\ will start a new line.
%% You may add \affil{} for affiliation, ie,
%\authors{Robert M. Groves\\
%\affil{Universitat de les Illes Balears}
%Floyd J. Fowler, Jr.\\
%\affil{University of New Mexico}
%}

%% Print Half Title and Title Page:
\halftitlepage
\titlepage


%%%%%%%%%%%%%%%%%%%%%%%%%%%%%%%%%%%%%%%%%%%%%%%%%%%%%%%%%%%%%%%%
%% Off Print Info

%% Add your info here:
\offprintinfo{PyFinances Requirements Document \\ Rev 1.0}{Jonathan A. Webb}
%\offprintinfo{Python3, Software Development for Engineers and Scientist,\\ First Edition.}{Jonathan A. Webb}

%% Can use \\ if title, and edition are too wide, ie,
%% \offprintinfo{Survey Methodology,\\ Second Edition}{Robert M. Groves}


%%%%%%%%%%%%%%%%%%%%%%%%%%%%%%%%%%%%%%%%%%%%%%%%%%%%%%%%%%%%%%%%
%% Copyright Page

%\begin{copyrightpage}{year}
%Title, etc
%\end{copyrightpage}

% Note, you must use \ to start indented lines, ie,
% 
% \begin{copyrightpage}{2004}
% Survey Methodology / Robert M. Groves . . . [et al.].
% \       p. cm.---(Wiley series in survey methodology)
% \    ``Wiley-Interscience."
% \    Includes bibliographical references and index.
% \    ISBN 0-471-48348-6 (pbk.)
% \    1. Surveys---Methodology.  2. Social 
% \  sciences---Research---Statistical methods.  I. Groves, Robert M.  II. %
% Series.\\

% HA31.2.S873 2004
% 001.4'33---dc22                                             2004044064
% \end{copyrightpage}

%%%%%%%%%%%%%%%%%%%%%%%%%%%%%%%%%%%%%%%%%%%%%%%%%%%%%%%%%%%%%%%%
%% Frontmatter >>>>>>>>>>>>>>>>

%%%%%%%%%%%%%%%%%%%%%%%%%%%%%%%%%%%%%%%%%%%%%%%%%%%%%%%%%%%%%%%%
%% Only Dedication (optional) 
%% or Contributor Page for edited books
%% before \tableofcontents

% ie,
%%%%%%%%%%%%%%%%%%%%%%%%%%%%%%%%%%%%%%%%%%%%%%%%%%%%%%%%%%%%%%%%
%  Contributors Page for Edited Book
%%%%%%%%%%%%%%%%%%%%%%%%%%%%%%%%%%%%%%%%%%%%%%%%%%%%%%%%%%%%%%%%

% If your book has chapters written by different authors,
% you'll need a Contributors page.

% Use \begin{contributors}...\end{contributors} and
% then enter each author with the \name{} command, followed
% by the affiliation information.

% \begin{contributors}
% \name{Masayki Abe,} Fujitsu Laboratories Ltd., Fujitsu Limited, Atsugi,
% Japan

% \name{L. A. Akers,} Center for Solid State Electronics Research, Arizona
% State University, Tempe, Arizona

% \name{G. H. Bernstein,} Department of Electrical and
% Computer Engineering, University of Notre Dame, Notre Dame, South Bend, 
% Indiana; formerly of
% Center for Solid State Electronics Research, Arizona
% State University, Tempe, Arizona 
% \end{contributors}

%%%%%%%%%%%%%%%%%%%%%%%%%%%%%%%%%%%%%%%%%%%%%%%%%%%%%%%%%%%%%%%%
\contentsinbrief %optional
\tableofcontents
% \listoffigures %optional
% \listoftables  %optional

%%%%%%%%%%%%%%%%%%%%%%%%%%%%%%%%%%%%%%%%%%%%%%%%%%%%%%%%%%%%%%%%
% Optional Foreword:

%\begin{foreword}
%text
%\end{foreword}

%%%%%%%%%%%%%%%%%%%%%%%%%%%%%%%%%%%%%%%%%%%%%%%%%%%%%%%%%%%%%%%%
% Optional Preface:

% ie,
% \begin{preface}
% This is an example preface.
% \prefaceauthor{R. K. Watts}
% \where{Durham, North Carolina\\
% September, 2004}

%%%%%%%%%%%%%%%%%%%%%%%%%%%%%%%%%%%%%%%%%%%%%%%%%%%%%%%%%%%%%%%%
% Optional Acknowledgments:

% \acknowledgments
% acknowledgment text
% \authorinitials{} % ie, I. R. S.


%%%%%%%%%%%%%%%%%%%%%%%%%%%%%%%%
%% Glossary Type of Environment:

% \begin{glossary}
% \term{<term>}{<description>}
% \end{glossary}

%%%%%%%%%%%%%%%%%%%%%%%%%%%%%%%%
% \begin{acronyms} 
% \acro{<term>}{<description>}
% \end{acronyms}

%%%%%%%%%%%%%%%%%%%%%%%%%%%%%%%%
%% In symbols environment <term> is expected to be in math mode; 
%% if not in math mode, use \term{\hbox{<term>}}

% \begin{symbols}
% \term{<math term>}{<description>}
% \term{\hbox{<non math term>}}Box used when not using a math symbol.
% \end{symbols}

%%%%%%%%%%%%%%%%%%%%%%%%%%%%%%%%
\begin{introduction}
%\introauthor{<name>}{<affil>}
This document serves as the requirements repository for the PyFinances computer code.  The 
PyFinances code will analyze an individuals spending and determine habits and make 
an estimate for the amount of money in an indivudals checking and savings account
as a function of time with estimates for statistical uncertainty.  This document
will contain the capability, performance and component requirements that guide
the development, and use of the software suite.
\end{introduction}

%%%%%%%%%%%%%%%%%%%%%%%%%%%%%%%%%%%%%%%%%%%%%%%%%%%%%%%%%%%%%%%%
%% End for Front Matter, Beginning of text of book  >>>>>>>>>>>

%% Short version of title without \\ may be written in sq. brackets:
 %% Optional Part :

% Chapter describing Level-I (i.e. Capability) requirements 
\chapter{Capability Requirements}

This chapter describes the high-level capability (Level-I) requirements that guide the 
development and use of the PyFinances software suite.  This doducment is divided into
Key Performance Parameters, Key Software Atributes, and Additional Performance Parameters.
Each term is defined below.
% ================================================================================ 
% ================================================================================ 

\begin{itemize}
    \item {\textbf{Key Performance Parameter (KPP)}} - A KPP is an software attribute 
that is considered critical to the development or use of the software capability.  If 
the final code does not adequately address KPPs then it will not be accepted by the 
end-user
    \item {\textbf{Key Software Attribute (KSA)}} - A KSA is a software capability 
considered crucial in achieving a balanced sokution to KPPs.  KSAs generaly cover
system maintainability or evolvability and are considered critical to end-user
acceptance of the software suite.
    \item {\textbf{Additional Performance Parameter (APP)}} - APAs are software
attributes that are deemed desirable, but not necessary for the development of,
or use of a software-suite. 
\end{itemize}

\section{Key Performance Parameters}
The following items are considered necessary capabilities that must be codified in the 
PyFinances software suite.
% ================================================================================ 
% ================================================================================ 

\subsection{KPP 1. Develop Spending Breakdown}
The PyFinances software suite {\textit{shall}} be capable of determining a day
by day spending profile for historical spending divided into categories of
{\textbf{gas}}, {\textbf{miscellaneous}}, {\textbf{groceries}}, 
{\textbf{restaurant}}, and {\textbf{bar}}.
% ================================================================================ 

\subsection{KPP 2. Analyze Historical Spending Trends}
The PyFinances software suite {\textit{shall}} be capable of analyzing historical
spending trends by the categories of {\textbf{gas}}, {\textbf{miscellaneous}}, 
{\textbf{groceries}}, {\texttt{restaurant}}, and {\textbf{bar}} spending.
% -------------------------------------------------------------------------------- 

\subsubsection{KPP 2.1. Develop Probability Distribution Functions}
The software {\textit{shall}} develop Probability Distribution Functions (PDF) for each 
spending category as well as total spending, subtracting the effects of bills, 
and planned expenses, as well as expected pay deductions.  
% -------------------------------------------------------------------------------- 

\subsubsection{KPP 2.2. Develop Cumulative Distribution Functions}
The software {\textit{shall}} develop Cumulative Distribution Functions (CDF) for
each spending category as well as total spending, subtracting the effects of bills, 
and planned expenses, as well as pay deductions.  
% ================================================================================
% ================================================================================ 

\subsection{KPP 3. Allocate Checking and Savings Accounts}
The software suite {\textit{shall}} allocate time dependent values for personal
wealth assuming a direct and predictable income.
% -------------------------------------------------------------------------------- 

\subsubsection{KPP 3.1. Checking Account}
The software {\textit{shall}} predict the day by day value of an individual 
checking account over the user defined time frame.
% -------------------------------------------------------------------------------- 

\subsubsection{KPP 3.2. Savings Account}
The software suite {\textit{shall}} predict the day by day value of an individual
savings account over the user defined time frame.
% ================================================================================ 
% ================================================================================ 

\subsection{KPP 4. Allocate Paycheck Info}
The software suite {\textit{shall}} determine all pay information necessary to
predict the day by day value of checking and savings accounts.
% -------------------------------------------------------------------------------- 

\subsubsection{KPP 4.1. Determine Pay Dates}
The software suite {\textit{shall}} be able to determine the date on which
paycheks are released for all reasonable pay allocation schemes.
% -------------------------------------------------------------------------------- 

\subsubsection{KPP 4.2. Determine Pay Deductions}
The software suite {\textit{shall}} be able to deduction salary from each paycheck
to account for medical and dental insurance, life insurance, taxes, 401k, and legal
support fees, as well as any other necessary deduction.
% ================================================================================ 
% ================================================================================ 

\subsection{KPP 5. Deduct Planned Expenses}
The software suite {\textit{shall}} be capable of deducting planned expenses
from the users portfolio on the user defined dates in order to support predictions
for the value of checking and savings account.
% ================================================================================ 
% ================================================================================ 

\subsection{KPP 6. Deduct Bills}
The software suite {\textit{shall}} be capable of deducting known bills from the
users checking and savings account on the desired dates in order to support
predictions for the value of each account. 
% ================================================================================ 
% ================================================================================ 

\subsection{KPP 7. Monte Carlo Method}
The software suite {\textit{shall}} calculate the mean values for checking and 
savings account as well as the plus and minus 2 standard deviation (2$\sigma$) 
using a Monte Carlo technique.
% ================================================================================
% ================================================================================ 
\section{Key Software Attributes}
The requirements in this section are focused on maintainability and 
evolvability.
% -------------------------------------------------------------------------------- 

\subsection{KSA 1. Testing}
The software {\textit{shall}} be built with unit tests and regression testing
for all primary functions, classes, and components.  Unit and regression
testing {\textit{shall}} be used to validate performance requirements, which
will validate all capability requirements that map to those performance requirements.
% -------------------------------------------------------------------------------- 

\subsection{KSA 2. Evolvability}
The software {\textit{shall}} be developed with object oriented and component
oriented practices for the purpose of ensuring that the software can be evolved
in the future to account for multiple bank accounts.
% ================================================================================ 
% ================================================================================ 

\section{Additional Performance Parameters}
The following attributes can increase the information that the user can extract
for their spending portfolios.  These attributes are considered advantageous but
not necessary.
% -------------------------------------------------------------------------------- 

\subsection{APP 1. Graphical Probability Distribution Functions}
The software suite {\textit{should}} allow a user to view their spending 
PDFs as .png documents
% -------------------------------------------------------------------------------- 

\subsection{APP 2. Graphical Cumulative Distribution Functions}
The software suire {\textit{should}} allow a user to view thei spending PDFs
% ================================================================================ 
% ================================================================================ 
% eof


% Chapter describing the Level-II (i.e. Performance) requirements
\chapter{Performance Requirements}

This chapter describes the level-II requirements that provide detail to 
how the software suite will meet the intent of the level-I requirements.
The verification of these level-II requirements will validate all level-I
requirements that they map to.
% ================================================================================ 
% ================================================================================ 

\section{Input Files}
The following input files are required for the operation of the PyFinances 
software suite.
% ================================================================================ 

\subsection{RunOptions File}
The program {\textit{shall}} use a file titled {\texttt{RunOptions.txt}} that
guides the exection of the software suite.  This file contains all information
necessary to guide the execution of the PyFinances computer program.
The file {\textit{shall}} contain the following user inputted variables;

\begin{enumerate}
\item {\textbf{Run Monte Carlo:}} True or False
\item {\textbf{Sample Size:}} Describes the number of Monte Carlo iterations 
used to determine a stochastic estimate of account values
\item {\textbf{Start Date:}} Describes the date where financial estimates should
beggin.
\item {\textbf{End Date:}} Describes the date after which financial estimates should
cease.
\item {\textbf{Checking Start Value:}}  The initial value of the checking account on
the {\textbf{Start Date}}.
\item {\textbf{Savings Start Value:}} The initial value of the savings account on
the {\textbf{Start Date}}.
\item {\textbf{Annual Salary:}} The expected annual salary, assuming a constant
pay stream.
\item{\textbf{Pay Frequency:}} The frequency that pay allocations are made.  The 
options {\textit{shall}} be 'weekly', 'two weeks', 'bi-monthly', and 'monthly'.
\item {\textbf{First Pay Date:}} The first date when a pay allocation is made.  This
date must be align with the 15th and the last day of the month  if 'bi-monthly' is
chosen, and the end of the month if 'monthly' is chosen.
\item {\textbf{Run Histogram:}} True or False
\item {\textbf{Bins:}} The number of bins used in the probability and cumulative
	               distribution functions.
\item {\textbf{Hist Start Date:}} The date within the historical spending trend 
that should be used as the start point for the development of a histoggram.
\item {\textbf{Hist End Date:}} The date within the historical spending trend
that should be used as the end point for the development of a histogram.
\item {\textbf{Daily Expense File:}} The name of the expense history file that will be 
used to generate a daily spending file to include the path-length to the file.
\item {\textbf{Deductions File:}} The name of the file .csv file containing pay
deduction information.
\item {\textbf{Expenses File:}} The name of the .csv file containing expected
expenses and account additions to include the path length.
\item {\textbf{Bills File:}} The name of the .csv file containing expected bill
information, to include the path length.
\end{enumerate}

If {\textbf{Run Monte Carlo}} is False, then {\textbf{Run Histogram}} must be 
True.  In this case, the {\textbf{bins}}, {\textbf{Hist Start Date}}, 
{\textbf{Hist End Date}} and {\textbf{Daily Expense File}} attributes must be 
populated.  If {\textbf{Run monte Carlo}} is True, then all fields must be 
populated.
% -------------------------------------------------------------------------------- 

\subsection{Daily Expenses File}
\label{sec:dailyexpenses}
The PyFinances software suite {\textit{shall} be capable of reading a 
file titled {\texttt{Daily\_Expenses.csv}}.  The {\texttt{Daily\_Expenses.csv}} file
contains a list of all expenses used to create a statistical spending profiles.  
The {\texttt{Daily\_Expenses.csv}} file has the following column headers.

\begin{enumerate}
    \item {\textbf{Date}} - The date of the transaction in the MM/DD/YY format
    \item {\textbf{Checking\_Debit}} - The amount subtracted from the checking account.
    \item {\textbf{Checking\_Addition}} - The amount added from the checking account.
    \item {\textbf{Savings\_Debit}} - The ammount subtracted from the savings account.
    \item {\textbf{Savings\_Addtion}} - The ammount added to the savings account.
    \item {\textbf{Expense\_Type}} - The type of expense, which can be one of the following
	                             entries, Bills, Misc, Groceries, Gas, Bar, 
				     Restaurant, Paycheck, Fed Taxes, State Taxes, 
				     Planned Expense.
    \item {\textbf{Vendor}} - The vendor from which the item was purchased or a 
	                      credit was given.
    \item {\textbf{Description}} - A description of the purchase or credit.
\end{enumerate}i

\noindent The {\texttt{Daily\_Expenses.csv}} file can contain multiple entries for 
the same day.
% -------------------------------------------------------------------------------- 

\subsection{Total Expenses File}
\label{sec:totalexpenses}
The PyFinances software {\textit{shall}} be capable of reading and writing a 
file titled the {\texttt{Total\_Expenses.csv}} file.  The {\texttt{Total\_Expenses.csv}}
file contains a breakdown for the total amount of money spend on each day for the 
following categories, which also act as colum headers.

\begin{enumerate}
    \item {\textbf{Date}} - The expense or credit date in the MM/DD/YY format
    \item {\textbf{Misc}} - Miscellaneous expenses not covered in other topics, 
	                    excluding taxes
    \item {\textbf{Bills}} - Money spend on bills
    \item {\textbf{Groceries}} - Montey spent on groceries
    \item {\textbf{Gas}} - Money spent on automative gasoline
    \item {\textbf{Bar}} - Any money spent specifically on alcohol
    \item {\textbf{Restaurant}} - Any money spent eating at restaurants
    \item {\textbf{Planned Expense}} - Any money spent on a topic that was planned.
	                               Christmas gifts are a good example.
\end{enumerate}
% --------------------------------------------------------------------------------

\subsection{Deductions File}
The PyFinances program {\textit{shall}} read a deductions file that contains 
all deductions from a paycheck, which can include federal and state taxes, 
401k, medical benefits and other information.
This file can be named whatever the user wishes it to be, but it must be defined
in the {\texttt{RunOptions.txt}} file under the name {\textbf{Deductions File}} and 
it must be a .csv file.  This file will contain the following headers.  The 
assumption made for these requirements is that the total paycheck allocation
will be allocated to the checking account, and any amount that should be 
placed in the savings account will be deducted as a bill.  In addition, it
is assumed that the money is deducted on the dates of pay allocation.

\begin{enumerate}
    \item {\textbf{Deduction}} - The financial amount to be deducted from the
	                         paycheck
    \item {\textbf{Description}} - A description of the deduction
\end{enumerate}
% --------------------------------------------------------------------------------

\subsection{Bills File}
The PyFinances program {\textit{shall}} read a bills file, which contains all
annual, usually monthly bills to be deduction from a checking and savings
account.
This file can be named whatever the user wishes it to be, but it must be defined
in the {\texttt{RunOptions.txt}} file under the name {\textbf{Bills File}} and
it must be a .csv file.  This file will contain the following headers.

\begin{enumerate}
    \item {\textbf{Date}} - The Date that bills will be deducted in the 
	                    format MM/DD/YYY.
    \item {\textbf{Checking\_Debit}} - The amount to be deducted from the 
	                               checking account.
    \item {\textbf{Checking\_Addition}} - The amount to be added to the 
	                                  checking account.
    \item {\textbf{Savings\_Debit}} - The amount to be deducted from the savings
	                              account
    \item {\textbf{Savings\_Addition}} - The amount to be added to the savings 
	                                 account 
    \item {\textbf{Description}} - A description of the transaction
\end{enumerate}
% --------------------------------------------------------------------------------

\subsection{Planned Expenses File}
The PyFinances program {\textit{shall}} read a Planned Expenses file, which 
contains all planned expenses that are not a paycheck deduction or a bill.
This file can be titled whatever the user wishes, but it must be defined in the
{\texttt{RunOptions}} file under the name {\textbf{Expenses File}} and it must be
a .csv file with the following headers.

\begin{enumerate}
    \item {\textbf{Date}} - The Date that bills will be deducted in the 
	                    format MM/DD/YYY.
    \item {\textbf{Checking\_Debit}} - The amount to be deducted from the 
	                               checking account.
    \item {\textbf{Checking\_Addition}} - The amount to be added to the 
	                                  checking account.
    \item {\textbf{Savings\_Debit}} - The amount to be deducted from the savings
	                              account
    \item {\textbf{Savings\_Addition}} - The amount to be added to the savings 
	                                 account 
    \item {\textbf{Description}} - A description of the transaction
\end{enumerate}
% --------------------------------------------------------------------------------

\subsection{PDF Files}
The PyFinances program {\textit{shall}} be capable of creating and reading a file
titled {\texttt{pdf\_data.csv}}, which contains the values of each probability 
distribution bin for each expense type.  The file should have a number of rows 
equal to the number of bins and the following header structure;

\begin{enumerate}
    \item {\textbf{bins}} - The bin number 
    \item {\textbf{groceries}} - The bin value for groceries
    \item {\textbf{misc}} - The bin value for miscellaneous
    \item {\textbf{bar}} - The bin value for bar
    \item {\textbf{restaurant}} - The bin value for restaurants
    \item {\textbf{gas}} - The bin value for gas
\end{enumerate}
% --------------------------------------------------------------------------------

\subsection{CDF Files}
The PyFinances program {\textit{shall}} be capable of creating and reading a file
titled {\texttt{cdf\_data.csv}}, which contains the values of each cumulative
distribution bin for each expense type.  The file should have a number of rows 
equal to the number of bins and the following header structure;

\begin{enumerate}
    \item {\textbf{bins}} - The bin number 
    \item {\textbf{groceries}} - The bin value for groceries
    \item {\textbf{misc}} - The bin value for miscellaneous
    \item {\textbf{bar}} - The bin value for bar
    \item {\textbf{restaurant}} - The bin value for restaurants
    \item {\textbf{gas}} - The bin value for gas
\end{enumerate}
% ================================================================================ 
% ================================================================================ 

\section{Pre-processor}
s section describes the requirments for the pre-processor and its subsequent
components
% ================================================================================ 

\subsection{Distributions Pre-processor}
The Distributions pre-processor shall read in the users historical spending
data and transform it into .csv files containing PDF and CDF information.  This
pre-processor {\textit{shall}} be invoked if the {\textbf{Run Histogram:}} option
is listed as True in the {\texttt{RunOptions.txt}} file.
% --------------------------------------------------------------------------------

\subsubsection{Read Daily\_Expenses.csv}
The distributions pre-processor {\textit{shall}} read
the {\texttt{Daily\_Expenses}} file described in Section 2.1.2.
This pre-processor.
% --------------------------------------------------------------------------------

\subsubsection{Create Total\_Expenses.csv}
The distributions pre-processor {\textit{shall}} transform the information 
within the {\texttt{Daily\_Expenses.csv}} file into the {\texttt{Total\_Expenses.csv}}
file described in Section 2.1.3.  This file should contain a day
by day account of how much money is spent in total for each of the categories to 
include {\texttt{bar}}, {\texttt{groceries}}, {\texttt{misc}}, {\texttt{gas}}, 
and {\texttt{restaurant}} as well as {\texttt{planned}}.
% --------------------------------------------------------------------------------

\subsubsection{Create pdf file}
The distributions pre-processor {\textit{shall}} read the data in the 
{\texttt{Total\_Expenses.csv}} file and transform it into the pdf file
described in Section 2.1.7.
% --------------------------------------------------------------------------------

\subsubsection{Create cdf file}
The distributions pre-processor {\textit{shall}} read the data in the 
{\texttt{cdf\_data.csv}} file and transform it into the cdf file
described in Section 2.1.8.
% --------------------------------------------------------------------------------

\subsubsection{Plot Data}
The distributions pre-processor {\textit{should}} produce a set of
pdf and cdf plots based on the data in the {\texttt{pdf\_data.csv}}
and the {\texttt{cdf\_data.csv}} file.
% --------------------------------------------------------------------------------

\subsection{Monte Carlo Pre-processor}
The Monte Carlo pre-processor {\textit{shall}} prepare all data necessary to run the 
Monte Carlo simluation, not already produced in Section 2.2.1.
% ================================================================================ 

\subsubsection{Validate Distributions}
The Monte Carlo pre-processor {\textit{shall}} verify that the {\texttt{cdf\_data.csv}}
file exists.  If the file does not exist, then the program {\textit{shall}} envoke
the Distributions Pre-processor to create the correct files.
% --------------------------------------------------------------------------------

\subsubsection{Create Date List}
The Monte Carlo pre-processor {\textit{shall}} produce a date list covering
every day from the {\textbf{Start Date:}} and {\textbf{End Date:}} described
in the {\texttt{RunOptions.txt}} file.  The dates within the list {\textit{shall}}
be in the format MM/DD/YYYY.
% --------------------------------------------------------------------------------

\subsubsection{Create Pay Date List}
The Monte Carlo pre-procssor {\textit{shall}} produce a list of
every pay date, where each date {\textit{shall}} be in the format MM/DD/YYYY.
% --------------------------------------------------------------------------------

\subsubsection{Create Input Containers}
The Monte Carlo pre-processor {\textit{shall}} produce a, or multiple 
containers to hold the information read from the {\texttt{bills.csv}}, 
{\texttt{deductions.csv}}, and {\texttt{planned\_expenses.csv}} files.
% ================================================================================ 
% ================================================================================ 
\section{Requirements Maping}
Table ~\ref{tab:table1} shows the mapping of level-I to level-II requirements.

\begin{table}[h!]
  \begin{center}
   \caption{Level-I to Level-II requirement mapping.}
    \label{tab:table1}
    \begin{tabular}{l|l}
      \textbf{Level-I Requirement} & \textbf{Level-II Requirements} \\
      \hline
      \hline
      \multirow{2}{*}{KPP 1. Develop Spending Breakdown} & 2.1.1. RunOptions file \\ 
      & 2.1.2. Daily Expense File  \\  & 2.1.3. Total Expenses File 
      \\ & 2.2.1. Distributions Pre-processor \\ & 2.2.1.1. Read Daily\_Expenses.csv \\
      & 2.2.1.2. Create Total Expenses.csv \\ & 2.2.1.3. Create pdf file \\ & 2.2.1.4. Create cdf file \\
      \hline
      \multirow{2}{*}{KPP 2. Analyze Historical Spending Trends} & 2.1.1. RunOptions file \\ 
      & 2.1.2. Daily Expense File  \\  & 2.1.3. Total Expenses File \\ & 2.2.1.3. Create pdf file \\
      & 2.2.1.4. Create cdf file \\
      \hline
      \multirow{2}{*}{KPP 2.1. Develop Probability Distribution Functions} & 2.1.1. RunOptions file \\ 
      & 2.1.7. PDF Files \\ & 2.2.1.3. Create pdf file \\
      \hline
      \multirow{2}{*}{KPP 2.2. Develop Cumulative Distribution Functions} & 2.1.1. RunOptions file \\ 
      & 2.1.7. CDF Files \\ & 2.2.1.4. Create cdf file \\
      \hline
      \multirow{2}{*}{KPP 3. Allocate Checking and Savings Accounts} & 2.1.1. RunOptions file \\ 
      & 2.2.2.4. Create Input Containers \\
      \hline
      \multirow{2}{*}{KPP 3.1. Checking Account} & 2.1.1. RunOptions file \\ 
      & 2.1.5. Bills File \\ & 2.1.6. Planned Expense File \\ & 2.2.4. Create Input Containers \\
      \hline
      \multirow{2}{*}{KPP 3.2. Savings Account} & 2.1.1. RunOptions file \\ 
      & 2.1.5. Bills File \\ & 2.1.6. Planned Expense File \\ & 2.2.4. Create Input Containers \\
      \hline
      \multirow{2}{*}{KPP 4. Allocate Paycheck Info} & 2.1.1. RunOptions file \\ 
      & 2.2.2.3. Create Pay Date List \\ & 2.2.2.4. Create Input Containers \\ 
      \hline
      \multirow{2}{*}{KPP 4.1. Determine Pay Dates} & 2.1.1. RunOptions file \\ 
      & 2.2.2.3. Create Pay Date List \\ 
      \hline
      \multirow{2}{*}{KPP 4.2 Determine Pay Deductions} & 2.1.1. RunOptions file \\ 
      & 2.1.4. Deductions File \\ & 2.2.2.4. Create Input Containers \\
      \hline
      \multirow{2}{*}{KPP 5. Deduct Planned Expenses} & 2.1.1. RunOptions file \\ 
      & 2.1.6. Planned Expenses File \\ & 2.2.2.4. Create Input Containers \\ 
      \hline
      \multirow{2}{*}{KPP 6. Deduct Bills} & 2.1.1. RunOptions file \\ 
      & 2.1.5. Bills File \\ & 2.2.2.4. Create Input Containers \\
      \hline
      \multirow{2}{*}{KPP 7. Monte Carlo Method} & 2.1.1. RunOptions file \\ 
      & 2.1.7. PDF Files \\ & 2.1.8. CDF Files \\
      \hline 
      KSA 1. Testing & Fill in \\
      \hline 
      KSA 2. Evolvability & Fill in \\
      \hline 
      \multirow{2}{*}{APP 1. Graphical Probability Distribution Function} & 2.1.1. RunOptions file \\ 
      & 2.1.7. PDF Files \\ & 2.2.1.5 Plot Data \\
      \hline
      \multirow{2}{*}{APP 2. Graphical Cumulative Distribution Function} & 2.1.1. RunOptions file \\ 
      & 2.1.7. CDF Files \\ & 2.2.1.5. Plot Data \\
      \hline
      \hline
    \end{tabular}
  \end{center}
\end{table}
% ================================================================================ 
% ================================================================================ 
% eof

%\begin{appendices}
%\input{Appendices/AppendixA.tex}
%\input{Appendices/AppendixB.tex}
%\end{appendices}
%\chapter[Statistical Processes]
%{Statistical Processes}
%\section{Temperature Measurements in a Circle}
%\subsection{Average}
%\subsection{Statistical Bias}
%\subsection{Geometric Bias}
%\subsection{Central Limit Theorum}
%\section{Distributions}
%\subsection{PDF}
%\subsection{CDF}
%\subsection{Sampling from Distributions}
%\subsection{Maxwellian Gas Distributions}
%\subsection{Linear Sampling}
%\subsection{Area Sampling}
%\subsection{Volume Sampling}

%%%%%%%%%%%%%%%%%%%%%%%%%%%%%%%%%%%%%%%%%%%%%%%%%%%%%%
%% optional prologue or prologues
% \chapter{Chapter Title}
% \prologue{<text>}{<author attribution>}

%%%%%%%%%%%%%%%%%%%%%%%%%%%%%%%%%%%%%%%%%%%%%
% Edited Book: Author and Affiliation
%%%%%%%%%%%%%%%%%%%%%%%%%%%%%%%%%%%%%%%%%%%%%

% After \chapter{Chapter Title}, you can
% enter the author name and embed the affiliation with
% \chapterauthors{(author name, or names)
% \chapteraffil{(affiliation or affiliations)}
% }    

% For instance:
% \chapter{Chapter Title}
% \chapterauthors{G. Alvarez and R. K. Watts
% \chapteraffil{Carnegie Mellon University, Pittsburgh, Pennsylvania}

% For separate affiliations you can use \affilmark{(number)} after
% the name of a particular author and before the matching affiliation:

% For instance:
% \chapter{Chapter Title}
% \chapterauthors{George Smeal, Ph.D.\affilmark{1}, Sally Smith,
% M.D.\affilmark{2}, and Stanley Kubrick\affilmark{1}
% \chapteraffil{\affilmark{1}AT\&T Bell Laboratories
% Murray Hill, New Jersey\\
% \affilmark{2}Harvard Medical School,
% Boston, Massachusetts}
% }

%%%%%%%%%%%%%%%%%%%%%%%

%% short version of section head, or one without \\ supplied in sq. brackets.

% \section[Introduction and fugue]{Introduction\\ and fugue}
% \subsection[This is the subsection]{This is the\\ subsection}
% \subsubsection{This is the subsubsection}
% \paragraph{This is the paragraph}

% \begin{chapreferences}{widest label}
% \bibitem{<label>}Reference
% \end{chapreferences}

% optional chapter bibliography using BibTeX,
% must also have \usepackage{chapterbib} before \begin{document}
% Must use root file with \include{chap1}, \include{chap2} form.
%\bibliographystyle{plain}
%\bibliography{<your .bib file name>}

% optional appendix at the end of a chapter:
% \chapappendix{<chap appendix title>}
% \chapappendix{} % no title

%%%%%%%%%%%%%%%%%%%%%%%%%%%%%%%%%%%%%%%%%%%%%%%%%%%%%%%%%%%%%%%%
%% End Matter >>>>>>>>>>>>>>>>>>

% \appendix{<optional title for appendix at end of book>}
% \appendix{} % appendix without title

% \begin{references}{<widest label>}
% \bibitem{sampref}Here is reference.
% \end{references}

%%%%%%%%%%%%%%%%%%%%%%%%%%%%%%%%%%%%%%%%%%%%%%%%%%%%%%%%%%%%%%%%
%% Optional Problem Sets: Can use this at the end of each chapter or at end
%% of book

% \begin{problems}
% \prob
% text

% \prob
% text

% \subprob
% text

% \subprob
% text

% \prob
% text
% \end{problems}

%%%%%%%%%%%%%%%%%%%%%%%%%%%%%%%%%%%%%%%%%%%%%%%%%%%%%%%%%%%%%%%%
%% Optional Exercises: Can use this at the end of each chapter or at end
%% of book

% \begin{exercises}
% \exer
% text

% \exer
% text

% \subexer
% text

% \subexer
% text

% \exer
% text
% \end{exercises}


%%%%%%%%%%%%%%%%%%%%%%%%%%%%%%%%%%%%%%%%%%%%%%%%%%%%%%%%%%%%%%%%
%% INDEX: Use only one index command set:

%% 1) The default LaTeX Index
\printindex

%% 2) For Topic index and Author index:

% \usepackage{multind}
% \makeindex{topic}
% \makeindex{authors}
% \begin{document}
% ...
% add index terms to your book, ie,
% \index{topic}{A term to go to the topic index}
% \index{authors}{Put this author in the author index}

%% (these are Wiley commands)
%\multiprintindex{topic}{Topic index}
%\multiprintindex{authors}{Author index}
\end{document}

%%%%%%% Demo of section head containing sample macro:
%% To get a macro to expand correctly in a section head, with upper and
%% lower case math, put the definition and set the box 
%% before \begin{document}, so that when it appears in the 
%% table of contents it will also work:

\newcommand{\VT}[1]{\ensuremath{{V_{T#1}}}}

%% use a box to expand the macro before we put it into the section head:

\newbox\sectsavebox
\setbox\sectsavebox=\hbox{\boldmath\VT{xyz}}

%%%%%%%%%%%%%%%%% End Demo
