\chapter{Capability Requirements}

This chapter describes the high-level capability (Level-I) requirements that guide the 
development and use of the PyFinances software suite.  This doducment is divided into
Key Performance Parameters, Key Software Atributes, and Additional Performance Parameters.
Each term is defined below.
% ================================================================================ 
% ================================================================================ 

\begin{itemize}
    \item {\textbf{Key Performance Parameter (KPP)}} - A KPP is an software attribute 
that is considered critical to the development or use of the software capability.  If 
the final code does not adequately address KPPs then it will not be accepted by the 
end-user
    \item {\textbf{Key Software Attribute (KSA)}} - A KSA is a software capability 
considered crucial in achieving a balanced sokution to KPPs.  KSAs generaly cover
system maintainability or evolvability and are considered critical to end-user
acceptance of the software suite.
    \item {\textbf{Additional Performance Parameter (APP)}} - APAs are software
attributes that are deemed desirable, but not necessary for the development of,
or use of a software-suite. 
\end{itemize}

\section{Key Performance Parameters}
The following items are considered necessary capabilities that must be codified in the 
PyFinances software suite.
% ================================================================================ 
% ================================================================================ 

\subsection{KPP 1. Develop Spending Breakdown}
The PyFinances software suite {\textit{shall}} be capable of determining a day
by day spending profile for historical spending divided into categories of
{\textbf{gas}}, {\textbf{miscellaneous}}, {\textbf{groceries}}, 
{\textbf{restaurant}}, and {\textbf{bar}}.
% ================================================================================ 

\subsection{KPP 2. Analyze Historical Spending Trends}
The PyFinances software suite {\textit{shall}} be capable of analyzing historical
spending trends by the categories of {\textbf{gas}}, {\textbf{miscellaneous}}, 
{\textbf{groceries}}, {\texttt{restaurant}}, and {\textbf{bar}} spending.
% -------------------------------------------------------------------------------- 

\subsubsection{KPP 2.1. Develop Probability Distribution Functions}
The software {\textit{shall}} develop Probability Distribution Functions (PDF) for each 
spending category as well as total spending, subtracting the effects of bills, 
and planned expenses, as well as expected pay deductions.  
% -------------------------------------------------------------------------------- 

\subsubsection{KPP 2.2. Develop Cumulative Distribution Functions}
The software {\textit{shall}} develop Cumulative Distribution Functions (CDF) for
each spending category as well as total spending, subtracting the effects of bills, 
and planned expenses, as well as pay deductions.  
% ================================================================================
% ================================================================================ 

\subsection{KPP 3. Allocate Checking and Savings Accounts}
The software suite {\textit{shall}} allocate time dependent values for personal
wealth assuming a direct and predictable income.
% -------------------------------------------------------------------------------- 

\subsubsection{KPP 3.1. Checking Account}
The software {\textit{shall}} predict the day by day value of an individual 
checking account over the user defined time frame.
% -------------------------------------------------------------------------------- 

\subsubsection{KPP 3.2. Savings Account}
The software suite {\textit{shall}} predict the day by day value of an individual
savings account over the user defined time frame.
% ================================================================================ 
% ================================================================================ 

\subsection{KPP 4. Allocate Paycheck Info}
The software suite {\textit{shall}} determine all pay information necessary to
predict the day by day value of checking and savings accounts.
% -------------------------------------------------------------------------------- 

\subsubsection{KPP 4.1. Determine Pay Dates}
The software suite {\textit{shall}} be able to determine the date on which
paycheks are released for all reasonable pay allocation schemes.
% -------------------------------------------------------------------------------- 

\subsubsection{KPP 4.2. Determine Pay Deductions}
The software suite {\textit{shall}} be able to deduction salary from each paycheck
to account for medical and dental insurance, life insurance, taxes, 401k, and legal
support fees, as well as any other necessary deduction.
% ================================================================================ 
% ================================================================================ 

\subsection{KPP 5. Deduct Planned Expenses}
The software suite {\textit{shall}} be capable of deducting planned expenses
from the users portfolio on the user defined dates in order to support predictions
for the value of checking and savings account.
% ================================================================================ 
% ================================================================================ 

\subsection{KPP 6. Deduct Bills}
The software suite {\textit{shall}} be capable of deducting known bills from the
users checking and savings account on the desired dates in order to support
predictions for the value of each account. 
% ================================================================================ 
% ================================================================================ 

\subsection{KPP 7. Monte Carlo Method}
The software suite {\textit{shall}} calculate the mean values for checking and 
savings account as well as the plus and minus 2 standard deviation (2$\sigma$) 
using a Monte Carlo technique.
% ================================================================================
% ================================================================================ 
\section{Key Software Attributes}
The requirements in this section are focused on maintainability and 
evolvability.
% -------------------------------------------------------------------------------- 

\subsection{KSA 1. Testing}
The software {\textit{shall}} be built with unit tests and regression testing
for all primary functions, classes, and components.  Unit and regression
testing {\textit{shall}} be used to validate performance requirements, which
will validate all capability requirements that map to those performance requirements.
% -------------------------------------------------------------------------------- 

\subsection{KSA 2. Evolvability}
The software {\textit{shall}} be developed with object oriented and component
oriented practices for the purpose of ensuring that the software can be evolved
in the future to account for multiple bank accounts.
% ================================================================================ 
% ================================================================================ 

\section{Additional Performance Parameters}
The following attributes can increase the information that the user can extract
for their spending portfolios.  These attributes are considered advantageous but
not necessary.
% -------------------------------------------------------------------------------- 

\subsection{APP 1. Graphical Probability Distribution Functions}
The software suite {\textit{should}} allow a user to view their spending 
PDFs as .png documents
% -------------------------------------------------------------------------------- 

\subsection{APP 2. Graphical Cumulative Distribution Functions}
The software suire {\textit{should}} allow a user to view thei spending PDFs
% ================================================================================ 
% ================================================================================ 
% eof
