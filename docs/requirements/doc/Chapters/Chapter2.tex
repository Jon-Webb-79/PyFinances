\chapter{Performance Requirements}

This chapter describes the level-II requirements that provide detail to 
how the software suite will meet the intent of the level-I requirements.
The verification of these level-II requirements will validate all level-I
requirements that they map to.
% ================================================================================ 
% ================================================================================ 

\section{Input Files}
The following input files are required for the operation of the PyFinances 
software suite.
% ================================================================================ 

\subsection{RunOptions File}
The program {\textit{shall}} use a file titled {\texttt{RunOptions.txt}} that
guides the exection of the software suite.  This file contains all information
necessary to guide the execution of the PyFinances computer program.
The file {\textit{shall}} contain the following user inputted variables;

\begin{enumerate}
\item {\textbf{Run Monte Carlo:}} True or False
\item {\textbf{Sample Size:}} Describes the number of Monte Carlo iterations 
used to determine a stochastic estimate of account values
\item {\textbf{Start Date:}} Describes the date where financial estimates should
beggin.
\item {\textbf{End Date:}} Describes the date after which financial estimates should
cease.
\item {\textbf{Checking Start Value:}}  The initial value of the checking account on
the {\textbf{Start Date}}.
\item {\textbf{Savings Start Value:}} The initial value of the savings account on
the {\textbf{Start Date}}.
\item {\textbf{Annual Salary:}} The expected annual salary, assuming a constant
pay stream.
\item{\textbf{Pay Frequency:}} The frequency that pay allocations are made.  The 
options {\textit{shall}} be 'weekly', 'two weeks', 'bi-monthly', and 'monthly'.
\item {\textbf{First Pay Date:}} The first date when a pay allocation is made.  This
date must be align with the 15th and the last day of the month  if 'bi-monthly' is
chosen, and the end of the month if 'monthly' is chosen.
\item {\textbf{Run Histogram:}} True or False
\item {\textbf{Bins:}} The number of bins used in the probability and cumulative
	               distribution functions.
\item {\textbf{Hist Start Date:}} The date within the historical spending trend 
that should be used as the start point for the development of a histoggram.
\item {\textbf{Hist End Date:}} The date within the historical spending trend
that should be used as the end point for the development of a histogram.
\item {\textbf{Daily Expense File:}} The name of the expense history file that will be 
used to generate a daily spending file to include the path-length to the file.
\item {\textbf{Deductions File:}} The name of the file .csv file containing pay
deduction information.
\item {\textbf{Expenses File:}} The name of the .csv file containing expected
expenses and account additions to include the path length.
\item {\textbf{Bills File:}} The name of the .csv file containing expected bill
information, to include the path length.
\end{enumerate}

If {\textbf{Run Monte Carlo}} is False, then {\textbf{Run Histogram}} must be 
True.  In this case, the {\textbf{bins}}, {\textbf{Hist Start Date}}, 
{\textbf{Hist End Date}} and {\textbf{Daily Expense File}} attributes must be 
populated.  If {\textbf{Run monte Carlo}} is True, then all fields must be 
populated.
% -------------------------------------------------------------------------------- 

\subsection{Daily Expenses File}
\label{sec:dailyexpenses}
The PyFinances software suite {\textit{shall} be capable of reading a 
file titled {\texttt{Daily\_Expenses.csv}}.  The {\texttt{Daily\_Expenses.csv}} file
contains a list of all expenses used to create a statistical spending profiles.  
The {\texttt{Daily\_Expenses.csv}} file has the following column headers.

\begin{enumerate}
    \item {\textbf{Date}} - The date of the transaction in the MM/DD/YY format
    \item {\textbf{Checking\_Debit}} - The amount subtracted from the checking account.
    \item {\textbf{Checking\_Addition}} - The amount added from the checking account.
    \item {\textbf{Savings\_Debit}} - The ammount subtracted from the savings account.
    \item {\textbf{Savings\_Addtion}} - The ammount added to the savings account.
    \item {\textbf{Expense\_Type}} - The type of expense, which can be one of the following
	                             entries, Bills, Misc, Groceries, Gas, Bar, 
				     Restaurant, Paycheck, Fed Taxes, State Taxes, 
				     Planned Expense.
    \item {\textbf{Vendor}} - The vendor from which the item was purchased or a 
	                      credit was given.
    \item {\textbf{Description}} - A description of the purchase or credit.
\end{enumerate}i

\noindent The {\texttt{Daily\_Expenses.csv}} file can contain multiple entries for 
the same day.
% -------------------------------------------------------------------------------- 

\subsection{Total Expenses File}
\label{sec:totalexpenses}
The PyFinances software {\textit{shall}} be capable of reading and writing a 
file titled the {\texttt{Total\_Expenses.csv}} file.  The {\texttt{Total\_Expenses.csv}}
file contains a breakdown for the total amount of money spend on each day for the 
following categories, which also act as colum headers.

\begin{enumerate}
    \item {\textbf{Date}} - The expense or credit date in the MM/DD/YY format
    \item {\textbf{Misc}} - Miscellaneous expenses not covered in other topics, 
	                    excluding taxes
    \item {\textbf{Bills}} - Money spend on bills
    \item {\textbf{Groceries}} - Montey spent on groceries
    \item {\textbf{Gas}} - Money spent on automative gasoline
    \item {\textbf{Bar}} - Any money spent specifically on alcohol
    \item {\textbf{Restaurant}} - Any money spent eating at restaurants
    \item {\textbf{Planned Expense}} - Any money spent on a topic that was planned.
	                               Christmas gifts are a good example.
\end{enumerate}
% --------------------------------------------------------------------------------

\subsection{Deductions File}
The PyFinances program {\textit{shall}} read a deductions file that contains 
all deductions from a paycheck, which can include federal and state taxes, 
401k, medical benefits and other information.
This file can be named whatever the user wishes it to be, but it must be defined
in the {\texttt{RunOptions.txt}} file under the name {\textbf{Deductions File}} and 
it must be a .csv file.  This file will contain the following headers.  The 
assumption made for these requirements is that the total paycheck allocation
will be allocated to the checking account, and any amount that should be 
placed in the savings account will be deducted as a bill.  In addition, it
is assumed that the money is deducted on the dates of pay allocation.

\begin{enumerate}
    \item {\textbf{Deduction}} - The financial amount to be deducted from the
	                         paycheck
    \item {\textbf{Description}} - A description of the deduction
\end{enumerate}
% --------------------------------------------------------------------------------

\subsection{Bills File}
The PyFinances program {\textit{shall}} read a bills file, which contains all
annual, usually monthly bills to be deduction from a checking and savings
account.
This file can be named whatever the user wishes it to be, but it must be defined
in the {\texttt{RunOptions.txt}} file under the name {\textbf{Bills File}} and
it must be a .csv file.  This file will contain the following headers.

\begin{enumerate}
    \item {\textbf{Date}} - The Date that bills will be deducted in the 
	                    format MM/DD/YYY.
    \item {\textbf{Checking\_Debit}} - The amount to be deducted from the 
	                               checking account.
    \item {\textbf{Checking\_Addition}} - The amount to be added to the 
	                                  checking account.
    \item {\textbf{Savings\_Debit}} - The amount to be deducted from the savings
	                              account
    \item {\textbf{Savings\_Addition}} - The amount to be added to the savings 
	                                 account 
    \item {\textbf{Description}} - A description of the transaction
\end{enumerate}
% --------------------------------------------------------------------------------

\subsection{Planned Expenses File}
The PyFinances program {\textit{shall}} read a Planned Expenses file, which 
contains all planned expenses that are not a paycheck deduction or a bill.
This file can be titled whatever the user wishes, but it must be defined in the
{\texttt{RunOptions}} file under the name {\textbf{Expenses File}} and it must be
a .csv file with the following headers.

\begin{enumerate}
    \item {\textbf{Date}} - The Date that bills will be deducted in the 
	                    format MM/DD/YYY.
    \item {\textbf{Checking\_Debit}} - The amount to be deducted from the 
	                               checking account.
    \item {\textbf{Checking\_Addition}} - The amount to be added to the 
	                                  checking account.
    \item {\textbf{Savings\_Debit}} - The amount to be deducted from the savings
	                              account
    \item {\textbf{Savings\_Addition}} - The amount to be added to the savings 
	                                 account 
    \item {\textbf{Description}} - A description of the transaction
\end{enumerate}
% --------------------------------------------------------------------------------

\subsection{PDF Files}
The PyFinances program {\textit{shall}} be capable of creating and reading a file
titled {\texttt{pdf\_data.csv}}, which contains the values of each probability 
distribution bin for each expense type.  The file should have a number of rows 
equal to the number of bins and the following header structure;

\begin{enumerate}
    \item {\textbf{bins}} - The bin number 
    \item {\textbf{groceries}} - The bin value for groceries
    \item {\textbf{misc}} - The bin value for miscellaneous
    \item {\textbf{bar}} - The bin value for bar
    \item {\textbf{restaurant}} - The bin value for restaurants
    \item {\textbf{gas}} - The bin value for gas
\end{enumerate}
% --------------------------------------------------------------------------------

\subsection{CDF Files}
The PyFinances program {\textit{shall}} be capable of creating and reading a file
titled {\texttt{cdf\_data.csv}}, which contains the values of each cumulative
distribution bin for each expense type.  The file should have a number of rows 
equal to the number of bins and the following header structure;

\begin{enumerate}
    \item {\textbf{bins}} - The bin number 
    \item {\textbf{groceries}} - The bin value for groceries
    \item {\textbf{misc}} - The bin value for miscellaneous
    \item {\textbf{bar}} - The bin value for bar
    \item {\textbf{restaurant}} - The bin value for restaurants
    \item {\textbf{gas}} - The bin value for gas
\end{enumerate}
% ================================================================================ 
% ================================================================================ 

\section{Pre-processor}
s section describes the requirments for the pre-processor and its subsequent
components
% ================================================================================ 

\subsection{Distributions Pre-processor}
The Distributions pre-processor shall read in the users historical spending
data and transform it into .csv files containing PDF and CDF information.  This
pre-processor {\textit{shall}} be invoked if the {\textbf{Run Histogram:}} option
is listed as True in the {\texttt{RunOptions.txt}} file.
% --------------------------------------------------------------------------------

\subsubsection{Read Daily\_Expenses.csv}
The distributions pre-processor {\textit{shall}} read
the {\texttt{Daily\_Expenses}} file described in Section 2.1.2.
This pre-processor.
% --------------------------------------------------------------------------------

\subsubsection{Create Total\_Expenses.csv}
The distributions pre-processor {\textit{shall}} transform the information 
within the {\texttt{Daily\_Expenses.csv}} file into the {\texttt{Total\_Expenses.csv}}
file described in Section 2.1.3.  This file should contain a day
by day account of how much money is spent in total for each of the categories to 
include {\texttt{bar}}, {\texttt{groceries}}, {\texttt{misc}}, {\texttt{gas}}, 
and {\texttt{restaurant}} as well as {\texttt{planned}}.
% --------------------------------------------------------------------------------

\subsubsection{Create pdf file}
The distributions pre-processor {\textit{shall}} read the data in the 
{\texttt{Total\_Expenses.csv}} file and transform it into the pdf file
described in Section 2.1.7.
% --------------------------------------------------------------------------------

\subsubsection{Create cdf file}
The distributions pre-processor {\textit{shall}} read the data in the 
{\texttt{cdf\_data.csv}} file and transform it into the cdf file
described in Section 2.1.8.
% --------------------------------------------------------------------------------

\subsubsection{Plot Data}
The distributions pre-processor {\textit{should}} produce a set of
pdf and cdf plots based on the data in the {\texttt{pdf\_data.csv}}
and the {\texttt{cdf\_data.csv}} file.
% --------------------------------------------------------------------------------

\subsection{Monte Carlo Pre-processor}
The Monte Carlo pre-processor {\textit{shall}} prepare all data necessary to run the 
Monte Carlo simluation, not already produced in Section 2.2.1.
% ================================================================================ 

\subsubsection{Validate Distributions}
The Monte Carlo pre-processor {\textit{shall}} verify that the {\texttt{cdf\_data.csv}}
file exists.  If the file does not exist, then the program {\textit{shall}} envoke
the Distributions Pre-processor to create the correct files.
% --------------------------------------------------------------------------------

\subsubsection{Create Date List}
The Monte Carlo pre-processor {\textit{shall}} produce a date list covering
every day from the {\textbf{Start Date:}} and {\textbf{End Date:}} described
in the {\texttt{RunOptions.txt}} file.  The dates within the list {\textit{shall}}
be in the format MM/DD/YYYY.
% --------------------------------------------------------------------------------

\subsubsection{Create Pay Date List}
The Monte Carlo pre-procssor {\textit{shall}} produce a list of
every pay date, where each date {\textit{shall}} be in the format MM/DD/YYYY.
% --------------------------------------------------------------------------------

\subsubsection{Create Input Containers}
The Monte Carlo pre-processor {\textit{shall}} produce a, or multiple 
containers to hold the information read from the {\texttt{bills.csv}}, 
{\texttt{deductions.csv}}, and {\texttt{planned\_expenses.csv}} files.
% ================================================================================ 
% ================================================================================ 
\section{Requirements Maping}
Table ~\ref{tab:table1} shows the mapping of level-I to level-II requirements.

\begin{table}[h!]
  \begin{center}
   \caption{Level-I to Level-II requirement mapping.}
    \label{tab:table1}
    \begin{tabular}{l|l}
      \textbf{Level-I Requirement} & \textbf{Level-II Requirements} \\
      \hline
      \hline
      \multirow{2}{*}{KPP 1. Develop Spending Breakdown} & 2.1.1. RunOptions file \\ 
      & 2.1.2. Daily Expense File  \\  & 2.1.3. Total Expenses File 
      \\ & 2.2.1. Distributions Pre-processor \\ & 2.2.1.1. Read Daily\_Expenses.csv \\
      & 2.2.1.2. Create Total Expenses.csv \\ & 2.2.1.3. Create pdf file \\ & 2.2.1.4. Create cdf file \\
      \hline
      \multirow{2}{*}{KPP 2. Analyze Historical Spending Trends} & 2.1.1. RunOptions file \\ 
      & 2.1.2. Daily Expense File  \\  & 2.1.3. Total Expenses File \\ & 2.2.1.3. Create pdf file \\
      & 2.2.1.4. Create cdf file \\
      \hline
      \multirow{2}{*}{KPP 2.1. Develop Probability Distribution Functions} & 2.1.1. RunOptions file \\ 
      & 2.1.7. PDF Files \\ & 2.2.1.3. Create pdf file \\
      \hline
      \multirow{2}{*}{KPP 2.2. Develop Cumulative Distribution Functions} & 2.1.1. RunOptions file \\ 
      & 2.1.7. CDF Files \\ & 2.2.1.4. Create cdf file \\
      \hline
      \multirow{2}{*}{KPP 3. Allocate Checking and Savings Accounts} & 2.1.1. RunOptions file \\ 
      & 2.2.2.4. Create Input Containers \\
      \hline
      \multirow{2}{*}{KPP 3.1. Checking Account} & 2.1.1. RunOptions file \\ 
      & 2.1.5. Bills File \\ & 2.1.6. Planned Expense File \\ & 2.2.4. Create Input Containers \\
      \hline
      \multirow{2}{*}{KPP 3.2. Savings Account} & 2.1.1. RunOptions file \\ 
      & 2.1.5. Bills File \\ & 2.1.6. Planned Expense File \\ & 2.2.4. Create Input Containers \\
      \hline
      \multirow{2}{*}{KPP 4. Allocate Paycheck Info} & 2.1.1. RunOptions file \\ 
      & 2.2.2.3. Create Pay Date List \\ & 2.2.2.4. Create Input Containers \\ 
      \hline
      \multirow{2}{*}{KPP 4.1. Determine Pay Dates} & 2.1.1. RunOptions file \\ 
      & 2.2.2.3. Create Pay Date List \\ 
      \hline
      \multirow{2}{*}{KPP 4.2 Determine Pay Deductions} & 2.1.1. RunOptions file \\ 
      & 2.1.4. Deductions File \\ & 2.2.2.4. Create Input Containers \\
      \hline
      \multirow{2}{*}{KPP 5. Deduct Planned Expenses} & 2.1.1. RunOptions file \\ 
      & 2.1.6. Planned Expenses File \\ & 2.2.2.4. Create Input Containers \\ 
      \hline
      \multirow{2}{*}{KPP 6. Deduct Bills} & 2.1.1. RunOptions file \\ 
      & 2.1.5. Bills File \\ & 2.2.2.4. Create Input Containers \\
      \hline
      \multirow{2}{*}{KPP 7. Monte Carlo Method} & 2.1.1. RunOptions file \\ 
      & 2.1.7. PDF Files \\ & 2.1.8. CDF Files \\
      \hline 
      KSA 1. Testing & Fill in \\
      \hline 
      KSA 2. Evolvability & Fill in \\
      \hline 
      \multirow{2}{*}{APP 1. Graphical Probability Distribution Function} & 2.1.1. RunOptions file \\ 
      & 2.1.7. PDF Files \\ & 2.2.1.5 Plot Data \\
      \hline
      \multirow{2}{*}{APP 2. Graphical Cumulative Distribution Function} & 2.1.1. RunOptions file \\ 
      & 2.1.7. CDF Files \\ & 2.2.1.5. Plot Data \\
      \hline
      \hline
    \end{tabular}
  \end{center}
\end{table}
% ================================================================================ 
% ================================================================================ 
% eof
